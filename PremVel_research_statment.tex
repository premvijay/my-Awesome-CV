\documentclass[12pt]{article}
\usepackage{geometry}
\geometry{a4paper, margin=1in}
\usepackage{amsmath, amssymb, hyperref}
\usepackage{enumitem}

\title{Research Statement}
\author{Premvijay Velmani}

\begin{document}

\maketitle
\section*{notes}
I have expertise in running cosmological simulations and analysing data from cosmological simulations from large collaborations like illustristng. I primarily look forward to continuing my research in understanding the astrophysical impact on dark matter haloes. but I can also work on studying other things related to dark matter distribution in the simulation, such as power spectrum dark matter haloes, filaments, etc. Also I can work on understanding galaxy halo connection, and in general galaxies and their evolution in cosmic web. In addition I can also apply for project positions more focussed on running mock simulations for large scale surveys like EUCLID, DESI, etc. 

My primary future research plan is to improve my time correlation analysis between astrophical processes of galaxies and the dark matter halo relaxation by exploring more relevant galaxy properties such as those associated with AGN feedbacks. Additionally, I would like to run numerical experiments with galaxy- halo like systems, for example, study how the orbits of dark matter test particles change in response to change in gravitational potential that mimics galaxy formation. I want to use that to build a physical model and predict the timescale expected for the dark matter at different halo-centric distances to respond to galactic feedbacks. Using this I will look for simple physics motivated relaxation models that may have explicit dependence on halo-centric distance and som timescale and see if that model can consistently explain the relaxation seen in realistic simulations such as IllustrisTNG. 

It will also be helpful if I run own set of cosmological and zoom simulations with hydrodynamics and prescriptions for galaxy formations.
Depending on the available resources I plan to run hydrodynamical simulations with some form of galaxy processes both cosmological and zoom-in. For example, I want to ask if different astrophysical processes were turned on and off for a period of time in the same simulation and study how that affects the dynamical evolution of the dark matter halo relaxation resonse.


\newpage
\section{Introduction}
\subsection{Current Research}
My research focuses on cosmology and large-scale structure to galactic astrophysics through both cosmological simulations with full galaxy formation prescription and other numerical experiments that semi-analytically models dark matter haloes with galaxies. Primarily, I study the dynamics and evolution of dark matter haloes within the cosmic web focusing on how these structures interact with the astrophysical processes associated with galaxy formation and evolution. In the ongoing research in cosmology and dark matter physics, the response of dark matter haloes to galaxies is frequently negleted or at best considered as nuissance parameters, modelled empirically based on simulations that are not without ambiguity. My current work helps in building a comprehensive physical model of this halo response that is one of the key ingredients in improving the analysis in those cosmological research. Additionally, by bridging galactic astrophysics with cosmology, this will also contribute to better understand the formation and evolution of galaxies, including their dynamical interaction with their host dark matter haloes.

\subsection{Research Expertise}
During the course of my PhD research, I have performed cosmological simulations including hydrodynamical ones that includes some of the baryonic astrophysical process such as cooling and star formation. I have primarily worked with a variety of large simulation particle data such as IllustrisTNG, EAGLE and CAMELS. I have also used structure finding codes to generate halo catalogues with merger trees and developed a kdtree based algorithm to match them between different simulations of same initial cosmological volumes. I have also developed codes to generate field information from simulation particle data, for both visualing and computing cosmologicalogical quantities such as the matter power spectrum and halo/galaxy properties such as mass profiles and shapes. I have performed various statistical techniques such as computing the correlation functions in multiple dimensions. 

I have developed analytical models of dark matter halo collapse along with the formation and evolution of galaxies and 

\section*{Current Research and Expertise}
My expertise includes running and analyzing large-scale cosmological simulations, such as those from the IllustrisTNG and EAGLE projects, as well as performing numerical experiments. I have also independently run simulations using GADGET4 and SWIFT, incorporating various baryonic physics and genrated halo catalogs and merger trees with structure finding tools such as ROCKSTAR and VELOCIraptor, as well as studying statistical properties like the matter power spectrum across different cosmologies and redshifts.

In my Ph.D. research, I developed a quasi-adiabatic relaxation framework to investigate how astrophysical feedback processes, such as those from active galactic nuclei (AGN) and stellar feedback, influence the internal dynamics of dark matter haloes over time. I also worked on extending self-similar models of galaxy formation to study how these processes might alter the structure of dark matter haloes at different stages of galaxy evolution. This work has demonstrated that time-dependent astrophysical processes have distinct, measurable impacts on halo properties, particularly on scales relevant to cosmology and galaxy formation studies.

\section*{Future Research Directions}
Building on my expertise in the astrophysical impacts on dark matter haloes, I plan to pursue several key avenues of research in a postdoctoral position:

\begin{enumerate}[label=\arabic*.]
    \item \textbf{Time-Correlation Analysis of Galaxy-Halo Interactions:} My immediate focus is on expanding time-correlation analyses to examine how specific galaxy properties—such as those associated with AGN feedback—impact the relaxation of dark matter haloes. Through this work, I aim to capture the transient and long-term effects of feedback processes, exploring how changes in galaxy mass distribution, star formation rates, and energy feedbacks alter the spatial and kinematic profiles of the surrounding dark matter.

    \item \textbf{Numerical Experiments on Galaxy-Halo System Dynamics:} I am particularly interested in conducting controlled numerical experiments with simplified galaxy-halo-like systems to track the orbital evolution of dark matter test particles. By simulating responses to varying gravitational potentials that mimic galaxy formation processes, I aim to model timescales for halo relaxation and the dynamic response at different halo-centric distances. These findings can contribute to more physically motivated models for the relaxation dynamics of haloes, incorporating both spatial dependencies and feedback-related timescales.

    \item \textbf{Hydrodynamical and Zoom Simulations for Galaxy-Halo Studies:} To support these goals, I also plan to run cosmological (zoom-in) simulations with full hydrodynamics and galaxy formation prescriptions. In addition to providing data on the large-scale structure, these simulations will enable me to validate theoretical models of galaxy-halo interactions and to study specific astrophysical impacts in finer detail than is possible in large-scale cosmological simulations alone.
\end{enumerate}

\section*{Additional Areas of Interest}
Beyond my primary research goals, I am eager to contribute to collaborative efforts that study other facets of dark matter distribution within simulations. This includes analyzing the power spectrum, halo-filament connectivity, and halo population statistics, especially as they relate to the distribution of galaxies in the cosmic web. I am also interested in roles that involve running mock simulations for current and upcoming large-scale surveys such as EUCLID and DESI, where insights from simulations can play a crucial role in interpreting observational data on galaxy clustering, gravitational lensing, and cosmic shear.

\section*{Conclusion}
I am enthusiastic about the opportunity to advance our understanding of galaxy-halo interactions through the lens of both cosmology and astrophysics. My background in running and analyzing cosmological simulations, combined with my theoretical knowledge of galaxy formation, positions me to make significant contributions to projects that bridge astrophysics and cosmology. I look forward to working within a collaborative environment where I can further develop my expertise in simulation-based cosmology and to contribute to our knowledge of the evolving universe.

\end{document}
