\documentclass[12pt]{article}
\usepackage{geometry}
\geometry{a4paper, margin=1in}
\usepackage{amsmath, amssymb, hyperref}
\usepackage{enumitem}

\title{Research Statement}
\author{Premvijay Velmani}

\begin{document}

\maketitle

\section*{Introduction}
My research focuses on cosmology and large-scale structure, specifically on understanding the dynamics and evolution of dark matter haloes within the cosmic web and how these structures interact with the astrophysical processes of galaxy formation and evolution. Through my work, I aim to build a comprehensive physical model of these interactions that bridges galaxy astrophysics with cosmology. By connecting dark matter halo responses to galaxy evolution, I seek to contribute to a more integrated picture of cosmic structure, aiding in the study of cosmological parameters and dark matter physics in simulations that reflect observed phenomena.

\section*{Current Research and Expertise}
My expertise includes running and analyzing large-scale cosmological simulations, such as those from the IllustrisTNG and EAGLE projects, as well as performing numerical experiments. I have also independently run simulations using GADGET4 and SWIFT, incorporating various baryonic physics and genrated halo catalogs and merger trees with structure finding tools such as ROCKSTAR and VELOCIraptor, as well as studying statistical properties like the matter power spectrum across different cosmologies and redshifts.

In my Ph.D. research, I developed a quasi-adiabatic relaxation framework to investigate how astrophysical feedback processes, such as those from active galactic nuclei (AGN) and stellar feedback, influence the internal dynamics of dark matter haloes over time. I also worked on extending self-similar models of galaxy formation to study how these processes might alter the structure of dark matter haloes at different stages of galaxy evolution. This work has demonstrated that time-dependent astrophysical processes have distinct, measurable impacts on halo properties, particularly on scales relevant to cosmology and galaxy formation studies.

\section*{Future Research Directions}
Building on my expertise in the astrophysical impacts on dark matter haloes, I plan to pursue several key avenues of research in a postdoctoral position:

\begin{enumerate}[label=\arabic*.]
    \item \textbf{Time-Correlation Analysis of Galaxy-Halo Interactions:} My immediate focus is on expanding time-correlation analyses to examine how specific galaxy properties—such as those associated with AGN feedback—impact the relaxation of dark matter haloes. Through this work, I aim to capture the transient and long-term effects of feedback processes, exploring how changes in galaxy mass distribution, star formation rates, and energy feedbacks alter the spatial and kinematic profiles of the surrounding dark matter.

    \item \textbf{Numerical Experiments on Galaxy-Halo System Dynamics:} I am particularly interested in conducting controlled numerical experiments with simplified galaxy-halo-like systems to track the orbital evolution of dark matter test particles. By simulating responses to varying gravitational potentials that mimic galaxy formation processes, I aim to model timescales for halo relaxation and the dynamic response at different halo-centric distances. These findings can contribute to more physically motivated models for the relaxation dynamics of haloes, incorporating both spatial dependencies and feedback-related timescales.

    \item \textbf{Hydrodynamical and Zoom Simulations for Galaxy-Halo Studies:} To support these goals, I also plan to run cosmological (zoom-in) simulations with full hydrodynamics and galaxy formation prescriptions. In addition to providing data on the large-scale structure, these simulations will enable me to validate theoretical models of galaxy-halo interactions and to study specific astrophysical impacts in finer detail than is possible in large-scale cosmological simulations alone.
\end{enumerate}

\section*{Additional Areas of Interest}
Beyond my primary research goals, I am eager to contribute to collaborative efforts that study other facets of dark matter distribution within simulations. This includes analyzing the power spectrum, halo-filament connectivity, and halo population statistics, especially as they relate to the distribution of galaxies in the cosmic web. I am also interested in roles that involve running mock simulations for current and upcoming large-scale surveys such as EUCLID and DESI, where insights from simulations can play a crucial role in interpreting observational data on galaxy clustering, gravitational lensing, and cosmic shear.

\section*{Conclusion}
I am enthusiastic about the opportunity to advance our understanding of galaxy-halo interactions through the lens of both cosmology and astrophysics. My background in running and analyzing cosmological simulations, combined with my theoretical knowledge of galaxy formation, positions me to make significant contributions to projects that bridge astrophysics and cosmology. I look forward to working within a collaborative environment where I can further develop my expertise in simulation-based cosmology and to contribute to our knowledge of the evolving universe.

\end{document}
