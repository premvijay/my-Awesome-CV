\documentclass[10pt]{article}
\usepackage[a4paper, margin=1in]{geometry}
\usepackage{aas_macros}
\usepackage{amsmath, amssymb, graphicx}
\usepackage{setspace}
% \usepackage[numbers]{natbib}
% \bibliographystyle{unsrtnat}
\usepackage[style=numeric,sorting=ynt]{biblatex}
\addbibresource{refers.bib}

\title{Research Statement}
\author{PremVijay Velmani}
\date{November, 2024}



\begin{document}


\maketitle
%\section*{notes}
%I have expertise in running cosmological simulations and analysing data from cosmological simulations from large collaborations like illustristng. I primarily look forward to continuing my research in understanding the astrophysical impact on dark matter haloes. but I can also work on studying other things related to dark matter distribution in the simulation, such as power spectrum dark matter haloes, filaments, etc. Also I can work on understanding galaxy halo connection, and in general galaxies and their evolution in cosmic web. In addition I can also apply for project positions more focussed on running mock simulations for large scale surveys like EUCLID, DESI, etc. 




%\newpage
\section{Overview}
\subsection{Current Research}
My research focuses on cosmology and large-scale structure to galactic astrophysics through both \textbf{cosmological simulations with full galaxy formation} prescription and more tractable numerical experiments of \textbf{dark matter haloes with galaxies}. Primarily, \textit{I study the dynamics and evolution of dark matter haloes within the cosmic web focusing on how these structures interact with the astrophysical processes associated with galaxy formation and evolution}. In the ongoing research in cosmology and dark matter physics, the response of dark matter haloes to galaxies is frequently neglected or, at best, considered as nuisance parameters, modelled empirically based on specific simulations. 
% Such models are also ambiguous and strongly depend on the current understanding and modelling of various baryonic astrophysics, which is expected to evolve further. 
My current work helps build a \textbf{comprehensive physical model of this halo response to galaxies} that is one of the key ingredients in ab initio modelling of the dark matter haloes. Additionally, by bridging galactic astrophysics with cosmology, this will also contribute to a better understanding of the formation and evolution of galaxies, including their dynamical interaction with their host dark matter haloes. In this regard, I have also obtained self-similar solutions to galaxy formation, interacting with its host dark matter halo.

\subsection{Proposal statement}
My primary research plan is to further understand and model the relaxation response of dark matter radial distribution to galaxies. Given that some of the key features have already been identified in my recent works \cite{2023MNRAS.520.2867V,2024arXiv240708030V,2024arXiv240804864V}, I expect to obtain a simple and realistic model of this with a few more measurements from cosmological simulations with galaxies. Further, by generalising the more tractable halo-galaxy systems \cite{2024JCAP...05..080V}, I expect to reproduce this relaxation response behaviour ab initio and provide a clear physical description for this model. 
% \cite{2022MNRAS.516.5849R}


% I will extend the analysis of halo relaxation through simulations and also employ more specialized cosmological simulations of galaxy formation by making use of the computational facilities at (*)  in collaboration. This is expected to help in modeling relaxation response accurately with significantly fewer parameters (see details below) 
% Extending the self-similar solutions and also through numerical experiments of controlled galaxy-halo systems, I expect to determine the exact physical mechanism behind relaxation. 

After establishing a simple, physical and accurate model for the response in the radial distribution, I would then try to also model response in the angular distribution (determining halo shapes) and velocity distribution, by following this approach of using a combination of simulations and other numerical experiments.

I also look forward to collaborating with other researchers at (*), in exploring various other interesting questions about the dark matter haloes, galaxies and other objects that build the large-scale structure of the Universe.

% Following this approach, I  to build a physical and accurate model for the response of the dark matter halo to galaxies beyond their radial distribution such as their angular distribution corresponding to the halo shapes and the velocity distributions. also look forward to study various other  aspects of dark halo response and in general the haloes, galaxies and other cosmological quantities through simulations.

\subsection{Research Expertise}

\begin{quote}
    \textit{With strong expertise in cosmological simulations with and without galaxy formation, I primarily look forward to doing both inference-based simulations to build better theoretical understanding and simulations-based inference in estimating interesting quantities from large-scale surveys.}
\end{quote}

During the course of my PhD research, I have performed cosmological simulations, including hydrodynamical ones that include some of the baryonic astrophysical processes such as cooling and star formation. I have primarily worked with various large simulation particle data such as IllustrisTNG, EAGLE, and CAMELS. I have also used structure finding codes to generate halo catalogues with merger trees and developed a kdtree-based algorithm to match them between different simulations of the same initial cosmological volumes. I have also developed codes to generate field information from simulation particle data, for both visualizing and computing cosmological quantities such as the matter power spectrum and halo/galaxy properties such as mass profiles and shapes. I have performed various statistical techniques, such as computing the correlation functions in multiple dimensions. 

I have obtained novel self-similar solutions of interacting dark matter halo and the formation and evolution of galaxy pseudo-disk. Extending the iterative mean field techniques employed in this approach, I also perform other numerical experiments in a more general case that can be directly compared against corresponding simulations.


%\section{Research plan}
%\subsection{Introduction}

%In my Ph.D. research, I developed a quasi-adiabatic relaxation framework to investigate how astrophysical feedback processes, such as those from active galactic nuclei (AGN) and stellar feedback, influence the internal dynamics of dark matter haloes over time. I also worked on extending self-similar models of galaxy formation to study how these processes might alter the structure of dark matter haloes at different stages of galaxy evolution. This work has demonstrated that time-dependent astrophysical processes have distinct, measurable impacts on halo properties, particularly on scales relevant to cosmology and galaxy formation studies.

\section{Key Contributions of My PhD Research}

In my PhD thesis titled, "Interplay of galaxy formation and dark matter halo evolution in the cosmic web", I have done the following research with Prof. Aseem Paranjape.
% on investigating the dynamical evolution of dark matter haloes and the impact of baryonic processes within them using large-scale hydrodynamical simulations such as IllustrisTNG, EAGLE, and CAMELS.

%\textbf{Key Contributions}
\begin{itemize}
    \item \textbf{Halo Relaxation Response in the cosmic web:} Through statistical analysis of a large number of haloes in cosmological simulations IllustrisTNG and EAGLE, we have developed models of quasi-adiabatic relaxation that accurately predicts the change in the dark matter radial distribution in response to the galaxy formation and evolution. These results also revealed the significance of feedback-related effects on the relaxation response and explicit dependence on halo-centric distance. This model provides an accurate fit to the relaxation responses observed in simulations of dark matter haloes and assists in the physical interpretation of the relaxation across a variety of haloes.
    
    \item \textbf{Role of Astrophysical processes and epoch:} Through an extensive collection of CAMELS simulations, we identified that not all but some of the simulation parameters controlling the feedback had a strong effect on the halo relaxation, and this is also significantly different at different redshifts.
    
    \item \textbf{Dynamics of the Relaxation:}  We have studied the evolutionary history of the relaxation of haloes across cosmic time along with its correlation with evolving halo and galaxy properties. This revealed that the relaxation response on the halo manifests immediately in the inner halo regions and with a time delay of around 2–3 Gyr in the outer halo regions, followed by periods of star formation activity. This explains the emergence of the dependence on the halo-centric distance at a given cosmic time.
    
    \item \textbf{Self-Similar Model for Halo-Galaxy Interplay:} We have obtained spherical self-similar solutions of mutually interacting dark matter halo and gas that radiatively cools and forms a pseudo galaxy disk with an artificial viscosity. With this more tractable approach to experiment and understand the relaxation mechanism, we also obtained relaxation response relations consistent with full simulations.
\end{itemize}


\section{Research Proposal}
Building on my expertise in the astrophysical impacts on dark matter haloes, I plan to pursue several key avenues of research in a postdoctoral position:

\begin{enumerate}
    \item \textbf{Time-Correlation Analysis of Galaxy-Halo Interactions:} My primary research plan is to explore the dynamics of the relaxation response further and extend the time-correlation analyses to examine how specific galaxy properties, such as those associated with AGN feedback, impact the relaxation of dark matter haloes. Through this work, I aim to capture the immediate and long-term effects of feedback processes, exploring how changes in galaxy mass distribution, star formation rates, and energy feedbacks affect the relaxation response of the dark matter halo at different halo-centric distances.

    \item \textbf{Numerical Experiments on Galaxy-Halo System Dynamics:} I am particularly interested in conducting controlled numerical experiments with simplified galaxy-halo-like systems to track the orbital evolution of dark matter test particles. By simulating responses to varying gravitational potentials that mimic galaxy formation processes, I aim to model timescales for halo relaxation and the dynamic response at different halo-centric distances. These findings can contribute to more physically motivated models for the relaxation dynamics of haloes, incorporating both spatial dependencies and feedback-related timescales.

%     My primary future research plan is to improve my time correlation analysis between astrophysical processes of galaxies and the dark matter halo relaxation by exploring more relevant galaxy properties such as those associated with AGN feedbacks. Additionally, I would like to run numerical experiments with galaxy- halo like systems, for example, study how the orbits of dark matter test particles change in response to change in gravitational potential that mimics galaxy formation. I want to use that to build a physical model and predict the timescale expected for the dark matter at different halo-centric distances to respond to galactic feedbacks. Using this I will look for simple physics motivated relaxation models that may have explicit dependence on halo-centric distance and som timescale and see if that model can consistently explain the relaxation seen in realistic simulations such as IllustrisTNG. 

% It will also be helpful if I run own set of cosmological and zoom simulations with hydrodynamics and prescriptions for galaxy formations.
% Depending on the available resources I plan to run hydrodynamical simulations with some form of galaxy processes both cosmological and zoom-in. For example, I want to ask if different astrophysical processes were turned on and off for a period of time in the same simulation and study how that affects the dynamical evolution of the dark matter halo relaxation response.

    \item \textbf{Hydrodynamical and Zoom Simulations for Galaxy-Halo Studies:} To support these goals, I also plan to run cosmological (zoom-in) simulations with full hydrodynamics and galaxy formation prescriptions. One of the interesting things to me is to perform simulations where the baryonic prescription is altered at specific timesteps during the course of the simulation and see how things respond to that change, especially the dark matter distribution in haloes.
    % In addition to providing data on the large-scale structure, these simulations will enable me to validate theoretical models of galaxy-halo interactions and to study specific astrophysical impacts in finer detail than is possible in large-scale cosmological simulations alone.
\end{enumerate}

\subsection{Additional Areas of Interest}
Beyond my primary research goals, I am eager to contribute to collaborative efforts that study other facets of dark matter distribution within simulations. This includes analyzing the power spectrum, halo-filament connectivity, and halo population statistics, especially as they relate to the distribution of galaxies in the cosmic web. I am also interested in roles that involve running mock simulations for current and upcoming large-scale surveys such as EUCLID and DESI, where insights from simulations can play a crucial role in interpreting observational data on galaxy clustering, gravitational lensing, and cosmic shear. I list below some of my long-term research plans.

% \section{Postdoctoral Research Plans}

% I plan to build on my PhD work by further probing the interaction between dark matter dynamics and baryonic processes, with the following goals:

% \subsection{Other long term research plans}

\textbf{1. Include halo shapes to physical model of relaxation:}  
I aim to build a physical and accurate of response in halo shapes to the formation and evolution of galaxies

\textbf{2. Investigating Halo Substructure Evolution:}  
I will study how baryonic processes affect subhalo dynamics and tidal stripping within massive dark matter haloes, particularly in the context of satellite galaxy evolution.

\textbf{3. Collaboration with Upcoming Surveys:}  
I will work on integrating simulation results with upcoming surveys such as LSST and Euclid, helping to bridge the gap between observational data and theoretical predictions of dark matter distribution and galaxy formation.

\textbf{4. Application to Baryonification Schemes:}  
Incorporating the knowledge gained from detailed simulations into efficient baryonification schemes for fast cosmological predictions.

\textbf{5. Exploring Alternative Dark Matter Models:}  
I also intend to explore how alternative dark matter models, such as self-interacting dark matter (SIDM), alter the relaxation properties of haloes and how these can be distinguished from standard cold dark matter (CDM) in both simulations and observations.



\section{Conclusion}
By combining my experience with large-scale cosmological simulations and theoretical modelling, I aim to advance our understanding of the role different baryonic astrophysics play in mediating the response of dark matter haloes to galaxies. I am enthusiastic about the opportunity to advance our understanding of galaxy-halo interactions through the lens of both cosmology and astrophysics. My background in running and analyzing cosmological simulations, with and without galactic astrophysics, positions me to start contributing to projects that bridge astrophysics and cosmology immediately. I look forward to working within a collaborative environment where I can further develop my expertise in simulation-based cosmology and galactic astrophysics and contribute to our knowledge of the evolving universe.

% In my phD thesis titled, "Interplay of galaxy formation and dark matter halo evolution in the cosmic web", I have worked on
%More specifically, this helps to decouple the influence of baryonic astrophysical effects from the radial distribution of dark matter in these haloes. 

% \bibliography{refers}
\printbibliography


\end{document}
