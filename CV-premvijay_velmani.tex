%%%%%%%%%%%%%%%%%%%%%%%%%%%%%%%%%%%%%%%%%
% Awesome Resume/CV
% XeLaTeX Template
% Version 1.2 (27/3/2017)
%
% This template has been downloaded from:
% http://www.LaTeXTemplates.com
%
% Original author:
% Claud D. Park (posquit0.bj@gmail.com) with modifications by 
% Vel (vel@latextemplates.com)
%
% License:
% CC BY-NC-SA 3.0 (http://creativecommons.org/licenses/by-nc-sa/3.0/)
%
% Important note:
% This template must be compiled with XeLaTeX, the below lines will ensure this
%!TEX TS-program = xelatex
%!TEX encoding = UTF-8 Unicode
%
%%%%%%%%%%%%%%%%%%%%%%%%%%%%%%%%%%%%%%%%%

%----------------------------------------------------------------------------------------
%	PACKAGES AND OTHER DOCUMENT CONFIGURATIONS
%----------------------------------------------------------------------------------------


\documentclass[11pt, a4paper]{awesome-cv} % A4 paper size by default, use 'letterpaper' for US letter
\definecolor{purple}{HTML}{2d1135}
\definecolor{lightpurple}{HTML}{fffaff}
\usepackage{pagecolor}
% \pagecolor{purple}

\geometry{left=1.9cm, top=.8cm, right=1.9cm, bottom=1.8cm, footskip=.5cm}
% \geometry{left=2cm, top=1.5cm, right=2cm, bottom=2cm, footskip=.5cm} % Configure page margins with geometry

\fontdir[fonts/] % Specify the location of the included fonts

%Color for highlights
\colorlet{awesome}{awesome-skyblue} % Default colors include: awesome-emerald, awesome-skyblue, awesome-red, awesome-pink, awesome-orange, awesome-nephritis, awesome-concrete, awesome-darknight
\definecolor{awesome}{HTML}{4fbdb0} % Uncomment if you would like to specify your own color

% Colors for text - uncomment and modify
% \definecolor{darktext}{HTML}{a3d126}
% \definecolor{text}{HTML}{e8d3d3}
% \definecolor{graytext}{HTML}{fce300}
% \definecolor{lighttext}{HTML}{b8ba5b}
% \definecolor{sectiondivider}{HTML}{a3c995}

% Uncomment if you would like to specify your own color
% \definecolor{awesome}{HTML}{CA63A8}

% Colors for text
% Uncomment if you would like to specify your own color
\definecolor{darktext}{HTML}{414141}
\definecolor{text}{HTML}{333333}
\definecolor{graytext}{HTML}{5D5D5D}
\definecolor{lighttext}{HTML}{888888}
\definecolor{sectiondivider}{HTML}{5D5D5D}

\setbool{acvSectionColorHighlight}{true}
\renewcommand{\acvHeaderSocialSep}{\quad\textbar\quad} % If you would like to change the social information separator from a pipe (|) to something else

%----------------------------------------------------------------------------------------
%	PERSONAL INFORMATION
%	Comment any of the lines below if they are not required
%----------------------------------------------------------------------------------------

\name{PremVijay}{Velmani}
\address{\textbf{Inter-University Centre for Astronomy and Astrophysics, Pune, India - 411007}}
\mobile{ (+91) 8056837468}

\email{premv@iucaa.in}
\linkedin{premvijay-v-5118b2a3}
%\skype{skypeid}
%\stackoverflow{SOid}{SOname}
%\twitter{@twit}
%\linkedin{linkedin name}
%\reddit{reddit account}
%\xing{xing name}
%\extrainfo{test} % Other text you want to include on this line

\position{Senior Research Fellow{\enskip\cdotp\enskip}PhD} % Your expertise/fields
%\quote{``Make the change that you want to see in the world."} % A quote or statement

\makecvfooter{\today}{PremVijay Velmani~~~·~~~Curriculum Vitae}{\thepage} % Specify the letter footer with 3 arguments: (<left>, <center>, <right>), leave any of these blank if they are not needed

%----------------------------------------------------------------------------------------

\begin{document}

\makecvheader % Print the header


%----------------------------------------------------------------------------------------
%	CV/RESUME CONTENT
%	Each section is imported separately, open each file in turn to modify content
%----------------------------------------------------------------------------------------
%----------------------------------------------------------------------------------------
%	SECTION TITLE
%----------------------------------------------------------------------------------------

\cvsection{Research}

% \cvsubsection{Interests}
% \cvsubsection{Primary Expertise}

\textbf
{Cosmology and Large Scale Structure - Dark matter haloes, their evolution and distribution in the cosmic web - Galaxy formation, feedbacks and their impacts on host haloes - Cosmological (hydrodynamical) simulations, Analytical halo formation models and numerical experiments - Self-similar haloes and galaxies.}

I primarily work on an interface between cosmology and astrophysics of galaxies, trying to understand and build a physical model of the impacts of galactic astrophysical processes on the dark matter haloes. This will not only make it easier to study cosmology and dark matter physics with haloes but also helps build a consistent picture of galaxy formation and evolution.
% \begin{itemize}
%     \item I run cosmological simulations and work with simulation data from large collaborations.
%     \item I also use semi-analytical models of dark matter haloes and galaxies do numerical experiments with such models.
%     \item I work on an interface between cosmology and astrophysics of galaxies, trying to understand and build a physical model of the impacts of galactic astrophysical processes on the dark matter haloes. This will not only make it easier to study cosmology and dark matter physics with haloes but also helps build a consistent picture of galaxy formation and evolution.
% \end{itemize}


\cvsubsection{Experiences}

%----------------------------------------------------------------------------------------
%	SECTION CONTENT
%----------------------------------------------------------------------------------------

\begin{cventries}

%------------------------------------------------
{
\cventry
{PI: Prof. Aseem Paranjape} % Job title
{PhD Research Fellow at IUCAA} % Organization
{Pune, India} % Location
{2020-present} % Date(s)
{ % Description(s) of tasks/responsibilities
\begin{cvitems}
\item {I have run N-body and hydrodynamics simulations of cosmological volumes using GADGET4 and SWIFT codes including some of the baryon astrophysics.}
\item {Using structure finding codes such as ROCKSTAR and VELOCIraptor, I have made halo catolgues and merger trees.}
\item {I have studied cosmological information such as matter power spectrum and various halo and galaxy properties in simulations.}
\item {Besides my own simulations, I have also worked on simulations produced by large collaborations such as IllustrisTNG and EAGLE.}
\item {I also worked on Self-similar models of halo formation and evolution along with galaxy formation.}
\item {I also worked with semi-analytical models of dark matter haloes and galaxies and did numerical experiments with such models.}
\item {In a mini project done with Prof. Hector marin, I have inferred cosmological parameters from eBOSS and mock DESI data.}
\item {I am currently working in a data science collaboration focused on using machine learning techniques in cosmological data compression and inference.}
\item {In another collobaration, I am working on the effect of supermassive black holes on the nature and the evolution of overall dark matter in the haloes.}
\end{cvitems}
}
}

%------------------------------------------------

{
\cventry
{PI: Prof. Sukanta Panda} % Job title
{MS Thesis at IISER Bhopal} % Organization
{Bhopal, India} % Location
{2018-2019} % Date(s)
{ % Description(s) of tasks/responsibilities
\begin{cvitems}
\item {I worked on cosmological perturbations in an anisotropic Bianchi type-I background and its evolution in a Bouncing model.}
\end{cvitems}
}
}

\end{cventries}


\cvsubsection{Publications}

\begin{cventries}

{
\cventry
{Premvijay Velmani, Aseem Paranjape}
{The quasi-adiabatic relaxation of haloes in the IllustrisTNG and EAGLE cosmological simulations}
{\url{https://doi.org/10.1093/mnras/stad297}}
{}
{
}
}

~\\[-8mm]
{
\cventry
{Sujatha Ramakrishnan, Premvijay Velmani}
{Properties beyond mass for unresolved haloes across redshift and cosmology using correlations with local halo environment}
{\url{https://doi.org/10.1093/mnras/stac2605}}
{}
{
% \begin{cvitems}
% \item 
% \end{cvitems}
}
}

~\\[-8mm]
{
\cventry
{Premvijay Velmani, Aseem Paranjape}
{A self-similar model of galaxy formation and dark halo relaxation}
{\url{https://doi.org/10.1088/1475-7516/2024/05/080}}
{}
{
% \begin{cvitems}
% \item 
% \end{cvitems}
}
}

~\\[-8mm]
{
\cventry
{Premvijay Velmani, Aseem Paranjape}
{Dynamics of the response of dark matter halo to galaxy evolution in IllustrisTNG}
{\url{https://doi.org/10.48550/arXiv.2407.08030}}
{}
{
% \begin{cvitems}
% \item 
% \end{cvitems}
}
}

~\\[-8mm]
{
\cventry
{Premvijay Velmani, Aseem Paranjape}
{The evolving role of astrophysical modelling in dark matter halo relaxation response}
{\url{https://doi.org//10.48550/arXiv.2408.04864}}
{}
{
% \begin{cvitems}
% \item 
% \end{cvitems}
}
}

\end{cventries}


\newpage
\cvsection{Conferences and Events}

\begin{cventries}

{
\cventry
{Summer School on Cosmology 2022}
{The Abdus Salam International Centre for Theoretical Physics}
{Trieste, Italy}
{July 2022}
{
\begin{cvitems}
\item {Presented a talk on "Impact Of Galaxy Formation On The Dark Matter Haloes In The Cosmic Web" at ICTP, an UNESCO organisation.}
\end{cvitems}
}
}

{
\cventry
{Largest Cosmological Surveys and Big Data Science 2023}
{International Centre for Theoretical Sciences}
{Bengaluru, India}
{April 2023}
{
\begin{cvitems}
\item {Worked on a mini project analysing redshift space distortion information from mock DESI data}
\item {Presented a talk on "Impact Of Galaxy Formation On The Dark Matter Haloes In The Cosmic Web" at ICTP, an UNESCO organisation.}
\end{cvitems}
}
}


{
\cventry
{Pune-Mumbai Cosmology and Astro-Particle Meeting - 2}
{Inter-University Centre for Astronomy and Astrophysics}
{Pune, India}
{February 2024}
{
\begin{cvitems}
\item {Presented a talk on "Interplay of baryonic galaxies and their host dark haloes - Insights from self-similar analysis"}
\end{cvitems}
}
}

{
\cventry
{Pune-Mumbai Cosmology and Astro-Particle Meeting - 3}
{Tata Institute of Fundamental Research}
{Mumbai, India}
{September 2024}
{
\begin{cvitems}
\item {Discussion focussed meeting}
\end{cvitems}
}
}

{
\cventry
{11th KIAS Workshop on Cosmology and Structure Formation}
{Korea Institute for advanced Study(KIAS)}
{Hilton yeongju, South Korea}
{October 2024}
{
\begin{cvitems}
\item {Presented a talk and a poster on "Interplay of galaxy formation and the evolution of dark matter haloes in the cosmic web - Dynamics of Relaxation".}
\end{cvitems}
}
}


\cvsubsection{Other talks}

{
\cventry
{Last Friday Talk}
{Indian Institute of Science Education and Research, Pune}
{Pune, India}
{Jan 2024}
{
\begin{cvitems}
\item {Presented a talk on "A self-similar model of galaxy 
formation and dark halo relaxation".}
\end{cvitems}
}
}

{
\cventry
{Last Friday Talk}
{Inter-University Centre for Astronomy and Astrophysics}
{Pune, India}
{November 2023}
{
\begin{cvitems}
\item {Presented a talk on "Impact Of Galaxy Formation On The Dark Matter Haloes In The Cosmic Web".}
\end{cvitems}
}
}

{
\cventry
{Last Friday Talk}
{Inter-University Centre for Astronomy and Astrophysics}
{Pune, India}
{March 2024}
{
\begin{cvitems}
\item {Presented a talk on "A self-similar model of galaxy 
formation and dark halo relaxation".}
\end{cvitems}
}
}

\end{cventries}
%----------------------------------------------------------------------------------------
%	SECTION TITLE
%----------------------------------------------------------------------------------------

\cvsection{Education}

%----------------------------------------------------------------------------------------
%	SECTION CONTENT
%----------------------------------------------------------------------------------------

\begin{cventries}
{
\cventry
{Doctor of Philosophy}
{Inter University Centre for Astronomy and Astrophysics (IUCAA affiliated to JNU)}
{Pune, Maharashtra}
{July 2019 - July 2024 }
{
\begin{cvitems}
\item Thesis title: Interplay of galaxy formation and the evolution of dark matter haloes in the cosmic web
\item Thesis advisor: Prof. Aseem Paranjape
\end{cvitems}
}
}

%------------------------------------------------
{
\cventry
{Bachelor of Science and Master of Science (BS-MS) Dual Degree} % Degree
{Indian Institute of Science Education and Research (IISER) Bhopal } % Institution
{ Bhopal, Madhya Pradesh} % Location
{ August 2014 - May 2019} % Date(s)
{ % Description(s) bullet points
\begin{cvitems}
\item Obtained a CPI/CGPA of 9.4 with major in Physics and a minor in Mathematics.
\item MS thesis: "Evolution of anisotropic perturbations in bouncing cosmology" under the guidance of Prof. Sukanta Panda.
		% \item {At the end of 4.5 years. CPI: \textbf{9.35/10.0}}
		% \item {In those nine semesters.   SPI:    \textbf{9.56,  8.90,  9.16,  8.63,  9.30,  9.65,  9.11,  9.70, 10.0  /10.0}}
		% \item {Received \textbf{O} grade in PHY 616: General Relativity, PHY 306: Statistical Mechanics, MTH 304: Metric Spaces and Topology, MTH 403: Real Analysis II and HSS 609: Logic.}
		% \item {\textbf{Courses Credited}: \\ - In the first two years, courses are as per BS MS course curricula of IISER Bhopal. See \href{http://acad.iiserb.ac.in/bsms_core.php}{\underline{here}}. \\ - From 3rd year onwards, Physics core courses are as per the curricula of Physics department major. See \href{http://acad.iiserb.ac.in/bsms_pro_phy.php}{{\underline{here}}.}}
		% \item {Elective course credited: \\ In \textbf{Maths} as minor: \\ \space - Real Analysis-I, Real Analysis-II, Complex Analysis, Metric Spaces and Topology, Advanced Linear Algebra, Differentiable Manifolds and Lie Groups. \\ In \textbf{Physics} as major: \\ \space - Thermal Physics, Numerical Methods and Programming, Quantum Field Theory-I, Quantum Field Theory-II, General Relativity,  Cosmology-I and Cosmology-II.}
\end{cvitems}
}
}


%------------------------------------------------

%----------------------------------------------------------------------------------------
%	SECTION CONTENT
%----------------------------------------------------------------------------------------

% \begin{cventries}

%------------------------------------------------
{
\cventry
{Class 12 - plus two} % Degree
{Tamil Nadu Board of Secondary Education} % Institution
{ Chennai, Tamil Nadu} % Location
{ Graduated March 2014} % Date(s)
{ % Description(s) bullet points
\begin{cvitems}
\item { \textbf{94.58} \%, Obtained \textbf{96.25} \% excluding language subjects.}
\end{cvitems}
}
}
%----------------------------------------------------------------------------------------
%	SECTION CONTENT
%----------------------------------------------------------------------------------------


% \begin{cventries}
{

%------------------------------------------------
\cventry
{Class 10 - SSLC} % Degree
{Tamil Nadu Board of Secondary Education} % Institution
{ Chennai, Tamil Nadu} % Location
{ Graduated April 2012} % Date(s)
{ % Description(s) bullet points
	\begin{cvitems}
		\item {\textbf{93.6} \%, Obtained \textbf{99.33} \% excluding language subjects and 100 \% in science.}
	\end{cvitems}
}

%------------------------------------------------
%------------------------------------------------
}
\end{cventries}

%----------------------------------------------------------------------------------------
%	SECTION TITLE
%----------------------------------------------------------------------------------------
\newpage
\cvsection{Awards and fellowships}

%------------------------------------------------

\begin{cvhonors}

%------------------------------------------------

\cvhonor
{Senior Research Fellowship} % Award
{Inter-University Centre for Astronomy and Astrophysics} % Event
{Pune, India} % Location
{Aug.2021} % Date(s)

%------------------------------------------------

\cvhonor
{Junior Research Fellowship} % Award
{Inter-University Centre for Astronomy and Astrophysics} % Event
{Pune, India} % Location
{Aug.2019} % Date(s)

%------------------------------------------------

\cvhonor
{Junior Research Fellowship} % Award
{Council of Scientific \& Industrial Research} % Event
{New Delhi, India} % Location
{Jan.2019} % Date(s)

%------------------------------------------------

\cvhonor
{INSPIRE Scholarship} % Award
{IISER, Department of Science \& Technology} % Event
{Bhopal, India} % Location
{Aug.2014} % Date(s)

%------------------------------------------------

\cvhonor
{Cash Prize Award} % Award
{Second rank in 12th board exam} % Event
{Chennai, India} % Location
{June.2014} % Date(s)

%------------------------------------------------

\cvhonor
{SSLC cash prize award} % Award
{Centum in 10th board exam} % Event
{Chennai, India} % Location
{April.2012} % Date(s)





%------------------------------------------------

\end{cvhonors}

\newpage
%----------------------------------------------------------------------------------------
%	SECTION TITLE
%----------------------------------------------------------------------------------------

\cvsection{Professional Skills}

\cvsubsection{Computer skills}

%----------------------------------------------------------------------------------------
%	SECTION CONTENT
%----------------------------------------------------------------------------------------

\begin{cvskills}

%------------------------------------------------
% {
\cvskill
{Programming} % Category
% {Python, Mathematica, IDL, }
{ Python, Bash, C, C++, IDL, Fortran, R, Wolfram, Matlab/Octave
% \break  \hspace{7.5cm} ~~ Other python libraries including vispy, vpython, pyopengl, pyopencl and GUI programming in PyQt5.
} % Skills
% }

\cvskill
{Python libraries}
{Includes numpy, scipy, pandas, astropy, cobaya, camb, colossus, casa, h5py, sklearn, matplotlib, seaborn, \break getdist, pyqtgraph, vispy, vpython, pyopengl, pyopencl, pycuda, conda, mpi4py and PyQt5}

\cvskill
{Simulation tools}
{GADGET4, SWIFT, ROCKSTAR, VELOCIraptor, MUSIC2-monofonIC}

\cvskill
{CAS}
{Mathematica (packages xTensor, xCoba, xPert, xPand), Maple, Python sympy}

\cvskill
{Operating system}
{Linux system administration, Bash, distros including Arch Linux, Ubuntu, Fedora, RHEL, SUSE, Centos; \break HPC clusters with PBS jobs scheduler and NFS storage; Windows administration, Powershell, WSL}

\cvskill
{Markup Languages}
{LaTeX, HTML, CSS, Markdown, MS Office/ Libreoffice}
%------------------------------------------------

\cvskill
{Other software} % Category
{Adobe Creative Cloud apps, DaVinci Resolve, Blender, Poser} % Skills

% %------------------------------------------------

% \cvskill
% {Linux:} % Category
% {Linux system administration, LibreOffice, Geogebra, Distros including Arch Linux, Fedora, SUSE, Ubuntu.} % Skills

%------------------------------------------------

\end{cvskills}

\cvsubsection{Teaching skills}

\begin{cvskills}

\cvskill
{Physics, Astronomy and Cosmology}
{}

\end{cvskills}
%----------------------------------------------------------------------------------------
%	SECTION TITLE
%----------------------------------------------------------------------------------------


\cvsection{Academic Achievements}


\begin{cvhonors}

%------------------------------------------------

\cvhonor
{Summer School on Cosmology 2022}
{Invited with funds by ICTP, UNESCO}
{Trieste, Italy}
{May 2022}



\cvhonor
{CSIR UGC NET - Physics} % Organization
{Qualified JRF with a score of 115 and all India rank of 116} % Job title
{India} % Location
{Jun 2018 } % Date(s)

\cvhonor
{JEE MAIN / AIEEE}
{Qualified with a score 190} % Job title
{India} % Location
{Aril 2014 } % Date(s)


%------------------------------------------------

\cvhonor
{NEET UG / AIPMT} % Organization
{Qualified with a score of 350.} % Job title
{India} % Location
{May 2014 } % Date(s)

%------------------------------------------------

\cvhonor
{JEE ADVANCED} % Organization
{Qualified with a rank of \textbf{4056} within OBC category.} % Job title
{India} % Location
{June 2014 } % Date(s)


\cvhonor
{SOLOLEARN - Python} % Organization
{Completed with \href{https://www.sololearn.com/Certificate/1073-553168/pdf/}{\underline{certification}}} % Job title
{} % Location
{June 2016} % Date(s)
% { % Description(s) of tasks/responsibilities
% \begin{cvitems}
% \item { Completed Python 3 course in Sololearn and the certificate is \href{https://www.sololearn.com/Certificate/1073-553168/pdf/}{\underline{here}}.}
% \end{cvitems}
% }

%------------------------------------------------
%------------------------------------------------

\end{cvhonors}

%----------------------------------------------------------------------------------------
%	SECTION TITLE
%----------------------------------------------------------------------------------------
% \newpage
\cvsection{Hobbies \& Interests}

%----------------------------------------------------------------------------------------
%	SECTION CONTENT
%----------------------------------------------------------------------------------------
% \cvsubsection{My philosophy}
% \begin{cvitems}
% \item {Beyond my primary research, I explore with various topics in both natural science and others such as history, politics, technology and sustainable development with a rational viewpoint.
% }
% \end{cvitems}

A strongly curiosity driven and challenge seeking person.

\cvsubsection{Reading}
I use the world wide web to explore various topics in both natural science and others such as history, politics, technology and sustainable development, with a rational viewpoint.

 % I explore various topics in science besides my expertise and such as history, politics to  Audiophile, 
\cvsubsection{Exploration} 
As a foodie, I travel to various places with special focus on trying different food.

\cvsubsection{Sports}
Primarily football, badminton, table tennis and swimming. 

\cvsubsection{Other personality}
I am a foodie, experimental cook, nutritionist, fitness freak, audiophile (also amateur singer), electrical/electronics hobbyist, amateur astronomer and computer gamer.

I enjoy doing lots of home experiments driven by my own curiosity. And I can relate with prof. R Feynman that ``Nearly everything is really interesting if you go into it deeply enough". 
I like challenges and I don't like doing simple things repetitively, so I keep exploring various new things. Trying to comprehend the logic behind complex things is my pleasure.


% \begin{cventries}

% %------------------------------------------------

% \cventry
% {Born curious, yes literally} % Role
% {Science:} % Event
% {        } % Location
% {        } % Date(s
% { % Description(s)
% \begin{cvitems}
% \item {Watching science documentaries and physics experiments videos on youtube, watching online lectures of my interests, reading advanced texts in Physics and Maths.}
% \item {Interpret science philosophically to get a deeper understanding of nature.}
% \item {Questioning everything and try to answer them with the help of Google, Quora and SE forums.}
% \end{cvitems}
% }

% %------------------------------------------------
% \cventry
% {Creative thinking and innovation is fun.} % Role
% {Technology:} % Event
% {        } % Location
% {        } % Date(s
% { % Description(s)
% \begin{cvitems}
% \item {Doing all kinds of geeky things like linux administration, programming and playing with electronics and electricals.}
% \end{cvitems}
% }

% %------------------------------------------------

% \cventry
% {My philosophy} % Role
% {Others:} % Event
% {        } % Location
% {        } % Date(s)
% { % Description(s)
% \begin{cvitems}
% \item {I like challenges and I don't like doing simple things repetitively, so I keep exploring various new things. Trying to comprehend the logic behind complex things is my pleasure.}
% \end{cvitems}
% }

% %------------------------------------------------

% \end{cventries}

% \newpage
%----------------------------------------------------------------------------------------
%	SECTION TITLE
%----------------------------------------------------------------------------------------

\cvsection{Extracurricular Activity}

%----------------------------------------------------------------------------------------
%	SECTION CONTENT
%----------------------------------------------------------------------------------------

\begin{cventries}

%------------------------------------------------

\cventry
{Presented a poster on basics of cosmology} % Affiliation/role
{IUCAA open science day} % Organization/group
{Pune, India} % Location
{Feb 2023} % Date(s)
{ % Description(s) of experience/contributions/knowledge
}

%------------------------------------------------
~\\[-5mm]
\cventry
{Created a video explaining probes of cosmology} % Affiliation/role
{IUCAA open science day} % Organization/group
{Pune, India} % Location
{Feb 2021} % Date(s)
{ % Description(s) of experience/contributions/knowledge
}

%------------------------------------------------
~\\[-5mm]
\cventry
{Demonstrated creation of gravitational waves} % Affiliation/role
{IUCAA open science day} % Organization/group
{Pune, India} % Location
{Feb 2019} % Date(s)
{ % Description(s) of experience/contributions/knowledge
}

%------------------------------------------------
~\\[-5mm]
\cventry
{Bad Ad Hoc Hypothesis on gravity at the cosmic horizon} % Affiliation/role
{Singularity-15, IISER Bhopal} % Organization/group
{ Bhopal, India} % Location
{ April 2015} % Date(s)
{ % Description(s) of experience/contributions/knowledge
% \begin{cvitems}
% \item {Presented a possible hypothesis to explain gravity at the horizons and defended it.}
% \end{cvitems}
}

%------------------------------------------------
~\\[-5mm]
\cventry
{Science Exhibition demonstration of a mega domino effect system} % Affiliation/role
{Singularity-16, IISER Bhopal} % Organization/group
{ Bhopal, India} % Location
{ September 2016} % Date(s)
{ % Description(s) of experience/contributions/knowledge
% \begin{cvitems}
% \item {Participated and constructed a domino effect of the size of a room with more than 10 different steps.}
% \end{cvitems}
}

%------------------------------------------------
~\\[-5mm]
\cventry
{Head of Artificial Intelligence and ML club; member of Physics, Mathematics and Astronomy Club} % Affiliation/role
{IISER Science Council} % Organization/group
{ Bhopal, India} % Location
{Aug 2014 - May 2019} % Date(s)
{ % Description(s) of experience/contributions/knowledge
	% Conducted Artificial Intelligence and ML club.
}

%------------------------------------------------
~\\[-5mm]
\cventry
{Workshop by spAts of IIT Kgp} % Affiliation/role
{Astronomy and Space Technology Awareness Camp} % Organization/group
{ Bhopal, India} % Location
{ April 2015} % Date(s)
{ % Description(s) of experience/contributions/knowledge
% \begin{cvitems}
% \item {Participated in workshop on "Astronomy and Space Science Technology" by spAts of IIT Kgp.}
% \end{cvitems}
}

%------------------------------------------------
~\\[-5mm]
\cventry
{National Science Quiz} % Affiliation/role
{Mimamsa 2015 by IISER Pune} % Organization/group
{ Mumbai, India} % Location
{ January 2015} % Date(s)
{ % Description(s) of experience/contributions/knowledge
% \begin{cvitems}
% \item {Participated in Mimamsa 2015 at Mumbai centre.}
% \end{cvitems}
}

%------------------------------------------------
~\\[-5mm]
\cventry
{National Science Quiz} % Affiliation/role
{Mimamsa 2016 by IISER Pune} % Organization/group
{ Bhopal, India} % Location
{ January 2016} % Date(s)
{ % Description(s) of experience/contributions/knowledge
% \begin{cvitems}
% \item {Participated in Mimamsa 2016 at Bhopal centre.}
% \end{cvitems}
}

%------------------------------------------------
~\\[-5mm]
\cventry
{Competitions} % Affiliation/role
{School level} % Organization/group
{Chennai, India} % Location
{ } % Date(s)
{ % Description(s) of experience/contributions/knowledge
\begin{cvitems}
\item {Winner in Science quiz intra school. Got selected in science talent exam conducted by The Hindu Educational Plus.}
\item {Winner in debate about Education System.}
\end{cvitems}
}

%------------------------------------------------

\end{cventries}

% %----------------------------------------------------------------------------------------
%	SECTION TITLE
%----------------------------------------------------------------------------------------

\cvsection{Languages}

%----------------------------------------------------------------------------------------
%	SECTION CONTENT
%----------------------------------------------------------------------------------------

\begin{cvskills}

%------------------------------------------------

\cvskill
{English:} % Category
{Fluent, Medium of education, communication at workplace} % Skills

%------------------------------------------------

\cvskill
{Tamil:} % Category
{Native, Mother Tongue} % Skills

%------------------------------------------------

\cvskill
{Hindi:} % Category
{Read or write but beginner level} % Skills

\cvskill
{Italian:}

%------------------------------------------------

\end{cvskills}

%----------------------------------------------------------------------------------------
%	SECTION TITLE
%----------------------------------------------------------------------------------------

\cvsection{References}

%----------------------------------------------------------------------------------------
%	SECTION CONTENT
%----------------------------------------------------------------------------------------

\begin{cvhonors}

%------------------------------------------------

\cvhonor
{Aseem Paranjape} % Position
{Associate Professor, IUCAA} % Committee
{Pune, India} % Location
{\textbf{  Prof.Dr.}} % Date(s)

%------------------------------------------------

%------------------------------------------------

\cvhonor
{Sukanta Panda} % Position
{Associate Professor, Department of Physics, IISER Bhopal} % Committee
{Bhopal, India} % Location
{\textbf{  Prof.Dr.}} % Date(s)

%------------------------------------------------

\end{cvhonors}


%----------------------------------------------------------------------------------------

\end{document}