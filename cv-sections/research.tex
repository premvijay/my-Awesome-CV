%----------------------------------------------------------------------------------------
%	SECTION TITLE
%----------------------------------------------------------------------------------------

\cvsection{Research}

% \cvsubsection{Interests}
\cvsubsection{Primary Expertise}

\textbf
{Cosmology and Large Scale Structure, Dark matter haloes, their distribution in the cosmic web and their evolution, along with their connection to Galaxy formation, Cosmological simulations, Analytical halo formation models such as self-similar collapse.}
I have expertise in cosmological simulations and semi-analytical models of dark matter haloes and galaxies. I work on an interface between cosmology and astrophysics of galaxies, trying to understand and build a physical model of the impacts of galactic astrophysical processes on the dark matter haloes. This will not only make it easier to study cosmology and dark matter physics with haloes  but also helps build a consistent picture of galaxy formation and evolution.



\cvsubsection{Experiences}

%----------------------------------------------------------------------------------------
%	SECTION CONTENT
%----------------------------------------------------------------------------------------

\begin{cventries}

%------------------------------------------------
{
\cventry
{PI: Prof. Aseem Paranjape} % Job title
{PhD Research Fellow at IUCAA} % Organization
{Pune, India} % Location
{2020-present} % Date(s)
{ % Description(s) of tasks/responsibilities
\begin{cvitems}
\item {I have run N-body and hydrodynamics simulations of cosmological volumes using GADGET4 and SWIFT codes including some of the baryon physics.}
\item {Using structure finding codes such as ROCKSTAR and VELOCIraptor, I have made halo catolgues and merger trees.}
\item {I have studied cosmological information such as matter power spectrum and various halo and galaxy properties in simulations.}
\item {Besides my own simulations, I have also worked on simulations produced by large collaborations such as IllustrisTNG and EAGLE.}
\item {I also work on Self-similar models of halo formation and evolution along with galaxy formation.}
\item {I have }
\end{cvitems}
}
}

%------------------------------------------------

{
\cventry
{PI: Prof. Sukanta Panda} % Job title
{MS Thesis at IISER Bhopal} % Organization
{Bhopal, India} % Location
{2018-2019} % Date(s)
{ % Description(s) of tasks/responsibilities
\begin{cvitems}
\item {I worked on cosmological perturbations in an anisotropic Bianchi type-I background and its evolution in a Bouncing model.}
\end{cvitems}
}
}

\end{cventries}


\cvsubsection{Publications}

\begin{cventries}

{
\cventry
{Premvijay Velmani, Aseem Paranjape}
{The quasi-adiabatic relaxation of haloes in the IllustrisTNG and EAGLE cosmological simulations}
{\url{https://doi.org/10.1093/mnras/stad297}}
{}
{
% \begin{cvitems}
% \item 
% \end{cvitems}
}
}

~\\[-8mm]
{
\cventry
{Sujatha Ramakrishnan, Premvijay Velmani}
{Properties beyond mass for unresolved haloes across redshift and cosmology using correlations with local halo environment}
{\url{https://doi.org/10.1093/mnras/stac2605}}
{}
{
% \begin{cvitems}
% \item 
% \end{cvitems}
}
}

~\\[-8mm]
{
\cventry
{Premvijay Velmani, Aseem Paranjape}
{A self-similar model of galaxy formation and dark halo relaxation}
{\url{https://doi.org/10.1088/1475-7516/2024/05/080}}
{}
{
% \begin{cvitems}
% \item 
% \end{cvitems}
}
}

~\\[-8mm]
{
\cventry
{Premvijay Velmani, Aseem Paranjape}
{Dynamics of the response of dark matter halo to galaxy evolution in IllustrisTNG}
{\url{https://doi.org/10.48550/arXiv.2407.08030}}
{}
{
% \begin{cvitems}
% \item 
% \end{cvitems}
}
}

~\\[-8mm]
{
\cventry
{Premvijay Velmani, Aseem Paranjape}
{The evolving role of astrophysical modelling in dark matter halo relaxation response}
{\url{https://doi.org//10.48550/arXiv.2408.04864}}
{}
{
% \begin{cvitems}
% \item 
% \end{cvitems}
}
}

\end{cventries}


% \newpage
\cvsubsection{Conferences and Events}

\begin{cventries}

{
\cventry
{Summer School on Cosmology 2022}
{The Abdus Salam International Centre for Theoretical Physics}
{Trieste, Italy}
{July 2022}
{
\begin{cvitems}
\item {Gave a talk on "Impact Of Galaxy Formation On The Dark Matter Haloes In The Cosmic Web" at ICTP, an UNESCO organisation.}
\end{cvitems}
}
}

{
\cventry
{Largest Cosmological Surveys and Big Data Science 2023}
{International Centre for Theoretical Sciences}
{Bengaluru, India}
{April 2023}
{
\begin{cvitems}
\item {Worked on a mini project analysing redshift space distortion information from mock DESI data}
\item {Gave a talk on "Impact Of Galaxy Formation On The Dark Matter Haloes In The Cosmic Web" at ICTP, an UNESCO organisation.}
\end{cvitems}
}
}

\end{cventries}